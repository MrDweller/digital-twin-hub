\documentclass[a4paper]{arrowhead}

\usepackage[yyyymmdd]{datetime}
\usepackage{etoolbox}
\usepackage[utf8]{inputenc}
\usepackage{multirow}
\usepackage{float}

\renewcommand{\dateseparator}{-}

\setlength{\parskip}{1em}

\newcommand{\fparam}[1]{\textit{\textcolor{ArrowheadBlue}{#1}}}

%% Special references
\newcommand{\fref}[1]{{\textcolor{ArrowheadBlue}{\hyperref[sec:functions:#1]{#1}}}}
\newcommand{\mref}[1]{{\textcolor{ArrowheadPurple}{\hyperref[sec:model:#1]{#1}}}}
\newcommand{\pdef}[1]{{\textcolor{ArrowheadGrey}{#1 \label{sec:model:primitives:#1} \label{sec:model:primitives:#1s}}}}
\newcommand{\pref}[1]{{\textcolor{ArrowheadGrey}{\hyperref[sec:model:primitives:#1]{#1}}}}

\newrobustcmd\fsubsection[5]{
  \addtocounter{subsection}{1}
  \addcontentsline{toc}{subsection}{\protect\numberline{\thesubsection}operation \textcolor{ArrowheadBlue}{#1}}
  \renewcommand*{\do}[1]{\rref{##1},\ }
  \subsection*{
    \thesubsection\quad
    #2 \textcolor{ArrowheadPurple}{#3} \\
    \small
    \hspace*{0.075\textwidth}\begin{minipage}{0.1\textwidth}
      \vspace*{1mm}
      Operation: \\
      \notblank{#4}{Input: \\}{}
      \notblank{#5}{Output: \\}{}
    \end{minipage}
    \begin{minipage}{0.825\textwidth}
      \vspace*{1mm}
      \textcolor{ArrowheadBlue}{#1} \\
      \notblank{#4}{\mref{#4} \\}{}
      \notblank{#5}{\mref{#5} \\}{}
    \end{minipage}
  }
  \label{sec:functions:#1}
}
\newrobustcmd\msubsection[2]{
  \addtocounter{subsection}{1}
  \addcontentsline{toc}{subsection}{\protect\numberline{\thesubsection}#1 \textcolor{ArrowheadPurple}{#2}}
  \subsection*{\thesubsection\quad#1 \textcolor{ArrowheadPurple}{#2}}
  \label{sec:model:#2} \label{sec:model:#2s}
}
\newrobustcmd\msubsubsection[3]{
  \addtocounter{subsubsection}{1}
  \addcontentsline{toc}{subsubsection}{\protect\numberline{\thesubsubsection}#1 \textcolor{ArrowheadPurple}{#2}}
  \subsubsection*{\thesubsubsection\quad#1 \textcolor{ArrowheadPurple}{#2}}
  \label{sec:model:#2} \label{sec:model:#2s}
}
%%

\begin{document}

%% Arrowhead Document Properties
\ArrowheadTitle{create-digital-twin HTTP/TLS/FORM} %e.g. ServiceDiscovery HTTP/TLS/JSON
\ArrowheadServiceID{login} % e.g. register
\ArrowheadType{Interface Design Description}
\ArrowheadTypeShort{IDD}
\ArrowheadVersion{4.6.1}
\ArrowheadDate{\today}
\ArrowheadAuthor{Jesper Frisk} % e.g Szvetlin Tanyi}
\ArrowheadStatus{RELEASE}
\ArrowheadContact{jesfri-8@student.ltu.se} % jerker.delsing@arrowhead.eu
\ArrowheadFooter{\href{www.arrowhead.eu}{www.arrowhead.eu}}
\ArrowheadSetup
%%

%% Front Page
\begin{center}
  \vspace*{1cm}
  \huge{\arrowtitle}

  \vspace*{0.2cm}
  \LARGE{\arrowtype}
  \vspace*{1cm}
\end{center}

%  \Large{Service ID: \textit{"\arrowid"}}
  \vspace*{\fill}

  % Front Page Image
  %\includegraphics{figures/TODO}

  \vspace*{1cm}
  \vspace*{\fill}

  % Front Page Abstract
  \begin{abstract}
    This document describes a HTTP protocol with TLS payload
    security and Form-data payload encoding variant of the \textbf{create-digital-twin} service.
  \end{abstract}
  \vspace*{1cm}

\newpage

%% Table of Contents
\tableofcontents
\newpage
%%

\section{Overview}
\label{sec:overview}

This document describes the \textbf{create-digital-twin} service interface, that provides the ability for system operators to create a new digital twin system. 
It’s implemented using protocol, encoding as stated in the following table:

\begin{table}[H]
  \centering
  \begin{tabular}{|l|l|l|l|}
    \rowcolor{gray!33} Profile type & Type & Version \\ \hline
    Transfer protocol & HTTP & 1.1 \\ \hline
    Data encryption & TLS & 1.3 \\ \hline
    Encoding & JSON & - \\ \hline
    Method & POST & - \\ \hline
  \end{tabular}
  \caption{Communication and semantics details used for the \textbf{create-digital-twin}
    service interface}
  \label{tab:comunication_semantics_profile}
\end{table}

This document provides the Interface Design Description IDD to the \textit{create-digital-twin -- Service Description} document.
For further details about how this service is meant to be used, please consult that document.

The rest of this document describes how to realize the \textbf{create-digital-twin} service HTTP/TLS/JSON interface in details.

\newpage

\section{Interface Description}
\label{sec:functions}

The service responses with the status code \texttt{200
  Ok} if called successfully. The error codes are, \texttt{400
  Bad Request} if request is malformed, \texttt{401 Unauthorized} if
improper client side certificate is provided.

\begin{lstlisting}[language=http,label={lst:create-digital-twin},caption={A \fref{create-digital-twin-service} invocation.}]
POST /create-digital-twin HTTP/1.1
{
  "controlCommands": [
    {
      "serviceDefinition": "string",
      "serviceUri": "string"
    }
  ],
  "physicalTwinConnection": {
    "connectionModel": {
      "address": "string",
      "port": 0
    },
    "connectionType": "string"
  },
  "sensedProperties": [
    {
      "intervalTime": 0,
      "sensorEndpointMode": "string",
      "serviceDefinition": "string",
      "serviceUri": "string"
    }
  ]
}
\end{lstlisting}

\begin{lstlisting}[language=http,label={lst:create-digital-twin},caption={A \fref{create-digital-twin-service} response.}]
{
  "address": "string",
  "authenticationInfo": "string",
  "port": 0,
  "systemName": "string"
}
\end{lstlisting}

\section{Data Models}
\label{sec:model}

Here, all data objects that can be part of the service calls associated with this service are listed in alphabetic order.
Note that each subsection, which describes one type of object, begins with the \textit{struct} keyword, which is meant to denote a JSON \pref{Object} that must contain certain fields, or names, with values conforming to explicitly named types.
As a complement to the primary types defined in this section, there is also a list of secondary types in Section \ref{sec:model:primitives}, which are used to represent things like hashes, identifiers and texts.

\msubsection{struct}{DigitalTwinRequest}
\label{sec:model:DigitalTwinRequest}
 
\begin{table}[H]
\begin{tabularx}{\textwidth}{| p{4cm} | p{4cm} | p{2cm} | X |} \hline
\rowcolor{gray!33} Field & Type & Mandatory & Description \\ \hline
controlCommands & \pref{List}$<$\hyperref[sec:model:ControlCommand]{ControlCommand}$>$ & no & Define control endpoints \\ \hline
physicalTwinConnection & \hyperref[sec:model:PhysicalTwinConnection]{PhysicalTwinConnection} & yes & Connection settings to the physical twin. \\ \hline
sensedProperties & \pref{List}$<$\hyperref[sec:model:SensedProperty]{SensedProperty}$>$ & no & Define sensor endpoints \\ \hline
\end{tabularx}
\end{table}

\msubsection{struct}{PhysicalTwinConnection}
\label{sec:model:PhysicalTwinConnection}
 
\begin{table}[H]
\begin{tabularx}{\textwidth}{| p{4.25cm} | p{3.5cm} | p{2cm} | X |} \hline
\rowcolor{gray!33} Field & Type & Mandatory & Description \\ \hline
connectionModel & \pref{ConnectionModel} & yes & Defines parameters with how the digital twin connects to the physical twin. \\ \hline
connectionType & \pref{String} & yes & Defines the type of connection that the digital twin uses to connect to the digital twin \\ \hline
\end{tabularx}
\end{table}

\msubsection{struct}{ConnectionModel}
\label{sec:model:ConnectionModel}
 
\begin{table}[H]
\begin{tabularx}{\textwidth}{| p{4.25cm} | p{3.5cm} | p{2cm} | X |} \hline
\rowcolor{gray!33} Field & Type & Mandatory & Description \\ \hline
address & \pref{Address} & yes & Network address. \\ \hline
port & \pref{PortNumber} & yes & Network port. \\ \hline
\end{tabularx}
\end{table}

\msubsection{struct}{ControlCommand}
\label{sec:model:ControlCommand}
 
\begin{table}[H]
\begin{tabularx}{\textwidth}{| p{4.25cm} | p{3.5cm} | p{2cm} | X |} \hline
\rowcolor{gray!33} Field & Type & Mandatory & Description \\ \hline
serviceDefinition & \pref{String} & yes & Definition of the service that will be registered that correspondes to the created endpoint. \\ \hline
serviceUri & \pref{String} & yes & Uri path for the created endpoint.\\ \hline
\end{tabularx}
\end{table}

\msubsection{struct}{SensedProperty}
\label{sec:model:SensedProperty}
 
\begin{table}[H]
\begin{tabularx}{\textwidth}{| p{4.25cm} | p{3.5cm} | p{2cm} | X |} \hline
\rowcolor{gray!33} Field & Type & Mandatory & Description \\ \hline
intervalTime & \pref{String} & no & If the "INTERVAL\_RETRIEVAL" sensor type is used then this defines the interval in seconds. \\ \hline
sensorEndpointMode & \pref{String} & yes & Defines the type of sensor endpoint that will be used by the digital twin.\\ \hline
serviceDefinition & \pref{String} & yes & Definition of the service that will be registered that correspondes to the created endpoint. \\ \hline
serviceUri & \pref{String} & yes & Uri path for the created endpoint.\\ \hline
\end{tabularx}
\end{table}

\msubsection{struct}{DigitalTwinResponse}
\label{sec:model:DigitalTwinResponse}
 
\begin{table}[H]
\begin{tabularx}{\textwidth}{| p{4cm} | p{4cm} | X |} \hline
\rowcolor{gray!33} Field & Type & Description \\ \hline
address & \pref{Address} & he created digital twins network address. \\ \hline
authenticationInfo &\pref{String}  & X.509 public key of the digital twin system. \\ \hline
port & \pref{PortNumber} & The created digital twins network port. \\ \hline
systemName & \pref{String} & Name of the created digital twin system.  \\ \hline
\end{tabularx}
\end{table}

\subsection{Primitives}
\label{sec:model:primitives}

As all messages are encoded using the JSON format \cite{bray2014json}, the following primitive constructs, part of that standard, become available.
Note that the official standard is defined in terms of parsing rules, while this list only concerns syntactic information. 

\begin{table}[ht!]
\begin{tabularx}{\textwidth}{| p{3cm} | X |} \hline
\rowcolor{gray!33} JSON Type & Description \\ \hline
\pdef{Value}                 & Any out of \pref{Object}, \pref{Array}, \pref{String}, \pref{Number}, \pref{Boolean} or \pref{Null}. \\ \hline
\pdef{Object}$<$A$>$         & An unordered collection of $[$\pref{String}: \pref{Value}$]$ pairs, where each \pref{Value} conforms to type A. \\ \hline
\pdef{Array}$<$A$>$          & An ordered collection of \pref{Value} elements, where each element conforms to type A. \\ \hline
\pdef{String}                & An arbitrary UTF-8 string. \\ \hline
\pdef{Number}                & Any IEEE 754 binary64 floating point number \cite{cowlishaw2019floating}, except for \textit{+Inf}, \textit{-Inf} and \textit{NaN}. \\ \hline
\pdef{Boolean}               & One out of \texttt{true} or \texttt{false}. \\ \hline
\pdef{Null}                  & Must be \texttt{null}. \\ \hline
\end{tabularx}
\end{table}

With these primitives now available, we proceed to define all the types specified in the \textbf{create-digital-twin} SD document without a direct equivalent among the JSON types.
Concretely, we define the \textbf{create-digital-twin} SD primitives either as \textit{aliases} or \textit{structs}.
An \textit{alias} is a renaming of an existing type, but with some further details about how it is intended to be used.
Structs are described in the beginning of the parent section.
The types are listed by name in alphabetical order.

\subsubsection{alias \pdef{Address} = \pref{String}}

A string representation of a network address. An address can be a version 4 IP address (RFC 791), a version 6 IP address (RFC 2460) or a DNS name (RFC 1034).

\subsubsection{alias \pdef{PortNumber} = \pref{Number}}

Decimal \pref{Number} in the range of 0-65535.

\subsubsection{alias \pdef{List}$<$A$>$ = \pref{Array}$<$A$>$}
There is no difference.

\section{Revision History}
\subsection{Amendments}

\noindent\begin{tabularx}{\textwidth}{| p{1cm} | p{3cm} | p{2cm} | X | p{4cm} |} \hline
\rowcolor{gray!33} No. & Date & Version & Subject of Amendments & Author \\ \hline

1 & YYYY-MM-DD & \arrowversion & & Xxx Yyy \\ \hline

\end{tabularx}

\subsection{Quality Assurance}

\noindent\begin{tabularx}{\textwidth}{| p{1cm} | p{3cm} | p{2cm} | X |} \hline
\rowcolor{gray!33} No. & Date & Version & Approved by \\ \hline

1 & YYYY-MM-DD & \arrowversion & Xxx Yyy \\ \hline

\end{tabularx}

\end{document}